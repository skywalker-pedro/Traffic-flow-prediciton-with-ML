
% >>>>>>>>>>>>>>>>>>
\slide{Motivação}{

\begin{itemize}
\item Necessidade de melhoria da capacidade dos aeroportos através da melhor utilização dos recursos disponíveis. 
\item Pista como principal gargalo no sistema. 
\item Soluções existentes possuem limitações.
\end{itemize}
}

\slide{Motivação}{
\begin{itemize}
    \item Ribeiro propõe uma solução  baseada na negociação entre aeronaves.
    \item Comportamento das aeronaves é baseado no Equilíbrio de Nash. 
    \item Se as aeronaves não chegarem a um acordo, a entidade aeroportuária intervem na negociação.
\end{itemize}
}

\slide{Motivação}{
O trabalho de Ribeiro possuí as seguintes limitações:
\begin{itemize}
    \item Pressupõe troca de informação perfeita entre as aeronaves.
    \item Modela os interesses das companhias aéreas e da Entidade Aeroportuária da mesma forma.
    \item As ofertas da negociação envolvem apenas deslocamentos na fila de espera. 
\end{itemize}
}

\slide{Proposta}{
\begin{itemize}
    \item Permitir ao agente realizar um lance junto da oferta de posição na fila.
    \item O comportamento dos agente passa a ser determinado por aprendizado por reforço. 
\end{itemize}
}
% >>>>>>>>>>>>>>>>>>
\slide{}{
\popup{Objetivos}{
\begin{itemize}
        \item Modelar o problema de Negociação de Slots como um Processo de Decisão de Markov
        \item Verificar o resultado do aprendizado na performance do agente treinado. 
        \item Avaliar o desempenho do sistema a partir do treinamento de diversos agentes.  
\end{itemize}
}
}

